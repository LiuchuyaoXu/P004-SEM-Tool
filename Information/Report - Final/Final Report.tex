\documentclass{article}
\usepackage{hyperref}
\usepackage{graphicx}
\usepackage[utf8]{inputenc}
\usepackage[margin=1in]{geometry}

\title{Real-time Diagnostic Tools for the Scanning Electron Microscope}
\author{Liuchuyao Xu\\Robinson College}
\begin{document}
\maketitle
\tableofcontents
\newpage

\section{Introduction}
\subsection{Project Objectives}
The scanning electron microscope (SEM) is a type of microscope that produces images using signals generated from the interaction between electrons and the surface under observation. Higher resolution can be achieved compared to the traditional optical microscope, since electrons have much lower wavelength than light. An SEM can have resolution lower than one nanometre, whereas that of an optical microscope is often limited to a few hundred nanometres. This has benefited a variety of fields. For example, scientists have been using the SEM to analyse the doping density in semiconductor \cite{SEM for semiconductor}. The aim of the project is to develop software tools in Python to support diagnosis of SEM images for the purpose of assisting the operator or automating procedures.

\subsection{Theory of the SEM}
TMR and additional explanation.

\subsection{How Fast Computing Can Aid SEM Operators}
Resolution, focus, aberrations, diagnosis, Fourier Transform, automating procedures.
Resolution and astigmatism are the most important factors to adjust.
Resolution and astigmatism are hard to judge.
Diagnostic information may help with judging resolution and astigmatism.
Diagnostic information may also allow automating procedures.

\section{The Algorithms}
\subsection{Histogram Equalisation}
The idea of histogram.
The maths of histogram.
Demonstration of the algorithm.
Speed test of the algorithm.

\subsection{Fast Fourier Transform} 
Idea of FT.
Maths of FFT.
Demonstration.
Speed test. 

\subsection{Focusing and Astigmatism Correction}
Idea.
Algorithm and flow chart.
Demonstration of current result.
Next step.

\section{The Software}
\subsection{Overview}
Overview of all modules.

\subsection{The SemImage Module}
Idea of the module.
Classes and functions.
Demo code.
Possible improvements.

\subsection{The SemTool Module}
Idea of the module.
Classes and functions.
Demo code.
Possible improvements.

\subsection{The SemCorrector Module}
Idea of the module.
Classes and functions.
Demo code.
Possible improvements.

\section{Demonstrations}
\subsection{Real-time Histogram Equalisation}
Screenshots of the software.

\subsection{Real-time Fast Fourier Transform}
Screenshots of the software.

\subsection{Automatic Focusing and Astigmatism Correction}
Screenshots of the software.
Test results.

\section{Next Steps}
Next steps.

\newpage
\begin{thebibliography}{}
    \bibitem{SEM wiki}
    "Scanning electron microscope." en.wikipedia.org. \url{https://en.wikipedia.org/wiki/Scanning_electron_microscope} (accessed May. 9, 2020).

    \bibitem{SEM for semiconductor}
    T. Agemura and T. Sekiguchi, "Secondary electron spectroscopy for imaging semiconductor materials," 2018 International Symposium on Semiconductor Manufacturing (ISSM), Tokyo, Japan, 2018, pp. 1-3, doi: 10.1109/ISSM.2018.8651171.
\end{thebibliography}
\end{document}
