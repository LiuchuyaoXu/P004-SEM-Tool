\documentclass[12pt,a4paper]{article}
\usepackage[utf8]{inputenc}
\usepackage[margin=1in]{geometry}
\usepackage[singlespacing]{setspace}
\usepackage{amsmath}
\usepackage{amsfonts}
\usepackage{amssymb}
\usepackage{authblk}
\usepackage{graphicx}
\graphicspath{{./Images/}}

\title{Real-time Diagnostic Tool for the Scanning Electron Microscope Technical Milestone Report}
\author{Liuchuyao Xu, Robinson College}
\affil{Supervisor: Dr David Holburn}

\begin{document}
\maketitle

\begin{abstract}
RESERVED LINE.

RESERVED LINE.

RESERVED LINE.

RESERVED LINE.

RESERVED LINE.
\end{abstract}

\section{Introduction}
\paragraph{}
The Scanning Electron Microscope (SEM) is an important technology for micro-surface observation. Conventional optical microscopes have resolutions limited to the wavelength of light (\textasciitilde 200 nm), whereas SEMs offer greater resolutions (\textasciitilde 0.5 nm).

\paragraph{}
SEMs work in a similar manner as optical microscopes. An electron beam is first generated and then shone onto the specimen. The reflected electrons are captured and used to produce an image. The resolution depend on the wavelength of the electrons, which in turn depends on the voltage used for accelerating the electrons.

\paragraph{}
The images produced by SEMs often have a variety of disturbances, such as lack of sharpness, low image quality, noises and image distortion and deformation. Sources of the disturbances could be interaction between the specimen and the electron beam, selection of observation conditions, specimen preparation or problems with the instrument itself.

\paragraph{}
Because of the disturbances, it is often not straightforward for an SEM operator to judge whether an image is good. The aim of the project is to develop a GUI-based software, which aids operators by providing real-time diagnostic information of the images. The initial requirement for the software is that it must produce results of Fast Fourier Transform (FFT) on the images at a minimum of 10 Hz. An SEM operator will constantly review the software, give feedback and come up with new features to be added.

\section{Progress}
\paragraph{}
\begin{figure}[h!]
  \centering
  \includegraphics[scale=0.8]{"GUI Screenshot"}
  \label{fig:GUI Screenshot}
  \caption{Preliminary GUI}
\end{figure}
Figure \ref{fig:GUI Screenshot} shows the preliminary GUI. The top left image is grabbed from the SEM and the bottom left figure shows the histogram of it. The two figures on the right are the same and show the result of FFT on the image.

\section{Next Steps}

\end{document}